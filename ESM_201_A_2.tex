% Options for packages loaded elsewhere
\PassOptionsToPackage{unicode}{hyperref}
\PassOptionsToPackage{hyphens}{url}
%
\documentclass[
]{article}
\usepackage{lmodern}
\usepackage{amssymb,amsmath}
\usepackage{ifxetex,ifluatex}
\ifnum 0\ifxetex 1\fi\ifluatex 1\fi=0 % if pdftex
  \usepackage[T1]{fontenc}
  \usepackage[utf8]{inputenc}
  \usepackage{textcomp} % provide euro and other symbols
\else % if luatex or xetex
  \usepackage{unicode-math}
  \defaultfontfeatures{Scale=MatchLowercase}
  \defaultfontfeatures[\rmfamily]{Ligatures=TeX,Scale=1}
\fi
% Use upquote if available, for straight quotes in verbatim environments
\IfFileExists{upquote.sty}{\usepackage{upquote}}{}
\IfFileExists{microtype.sty}{% use microtype if available
  \usepackage[]{microtype}
  \UseMicrotypeSet[protrusion]{basicmath} % disable protrusion for tt fonts
}{}
\makeatletter
\@ifundefined{KOMAClassName}{% if non-KOMA class
  \IfFileExists{parskip.sty}{%
    \usepackage{parskip}
  }{% else
    \setlength{\parindent}{0pt}
    \setlength{\parskip}{6pt plus 2pt minus 1pt}}
}{% if KOMA class
  \KOMAoptions{parskip=half}}
\makeatother
\usepackage{xcolor}
\IfFileExists{xurl.sty}{\usepackage{xurl}}{} % add URL line breaks if available
\IfFileExists{bookmark.sty}{\usepackage{bookmark}}{\usepackage{hyperref}}
\hypersetup{
  pdftitle={ESM\_201\_A\_2},
  hidelinks,
  pdfcreator={LaTeX via pandoc}}
\urlstyle{same} % disable monospaced font for URLs
\usepackage[margin=1in]{geometry}
\usepackage{graphicx,grffile}
\makeatletter
\def\maxwidth{\ifdim\Gin@nat@width>\linewidth\linewidth\else\Gin@nat@width\fi}
\def\maxheight{\ifdim\Gin@nat@height>\textheight\textheight\else\Gin@nat@height\fi}
\makeatother
% Scale images if necessary, so that they will not overflow the page
% margins by default, and it is still possible to overwrite the defaults
% using explicit options in \includegraphics[width, height, ...]{}
\setkeys{Gin}{width=\maxwidth,height=\maxheight,keepaspectratio}
% Set default figure placement to htbp
\makeatletter
\def\fps@figure{htbp}
\makeatother
\setlength{\emergencystretch}{3em} % prevent overfull lines
\providecommand{\tightlist}{%
  \setlength{\itemsep}{0pt}\setlength{\parskip}{0pt}}
\setcounter{secnumdepth}{-\maxdimen} % remove section numbering

\title{ESM\_201\_A\_2}
\author{}
\date{\vspace{-2.5em}}

\begin{document}
\maketitle

1a)

\includegraphics{ESM_201_A_2_files/figure-latex/unnamed-chunk-1-1.pdf}

\#1b) According to the graphs, it looks as if Barley, Oats, and Corn are
Linear, Lower Plateau(LLP). However, Sorghum looks like it could be
Linear, upper plateau(LUP) or maybe even Linear piecewise (PW) statistic
model.

\includegraphics{ESM_201_A_2_files/figure-latex/unnamed-chunk-2-1.pdf}

\#2a)

\includegraphics{ESM_201_A_2_files/figure-latex/unnamed-chunk-3-1.pdf}

2b) Observations: Corn and Wheat have more Nitrogen fertilizer than the
other fertilizers. Additionally, Corn has significantly more nitrogen
use than any of the other two crops. Soybeans has more potassium than
nitrogen and phosphate.

For Corn, the limiting nutrient is Nitrogen because it is needed the
most for the yields to grow. For Soybeans, the limiting nutrient is
Potassium. For Wheat, the limiting nutrient is Nitrogen as well.

3a)

\includegraphics{ESM_201_A_2_files/figure-latex/unnamed-chunk-4-1.pdf}

3b) Nitrogen fertilizer use is needed in high amounts of 160+ kg per
hectare to get to more than 400 Bushels per hectare of corn yield.
However, you do not need that much Phosphate and Potassium to reach the
same amount of yields. As you increase yield, you do not need additional
Phosphate and Potassium. But, you do need a lot more Nitrogen as you
want to increase yields.

The relationship is not linear. Although it does yield does increase in
the beginning as you add fertilizer, it tapers out. It no longer
provides high yields when you add more fertilizer.

\end{document}
